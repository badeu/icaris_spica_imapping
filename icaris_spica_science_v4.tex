% Template for ApJ-type papers

\documentclass[12pt,preprint]
%{aastex}
{emulateapj}

\usepackage{rotating}
\usepackage{amssymb,amsmath}
\usepackage{graphicx}[subfloat]
\usepackage{multirow}
\citestyle{aa}

\bibliographystyle{apj_w_etal}

\newcommand{\vect}[1]{\mathbf{#1}}
\newcommand{\xad}{\vect{x}}

\newcommand{\vdag}{(v)^\dagger}
\newcommand{\myemail}{jaguirre@nrao.edu}
\newcommand{\lsim}{{_{<}\atop^{\sim}}}
\newcommand{\gsim}{{_{>}\atop^{\sim}}}
\newcommand{\etal}{{et al.\/}}
\newcommand{\ie}{{\em ie.\/}}
\newcommand{\cmq}{cm{$^{-3}$}}
\newcommand{\per}{$^{\rm{-1}}$}
\newcommand{\tc}{{$\theta^1$~Orionis~C}}
%\newcommand{\msol}{M{$_{\odot}$}}
\def\msol{\ifmmode {\>M_\odot}\else {$M_\odot$}\fi}
\newcommand{\lsol}{L{$_{\odot}$}}
\newcommand{\kms}{km~s{$^{-1}$}}
\newcommand{\hii}{H~{\sc ii}}
\newcommand{\Hii}{H~{\sc ii}}
\newcommand{\Ha}{\mbox{H$\alpha$}}
\newcommand{\sii}{S~{\sc ii}}
\newcommand{\Feii}{Fe~{\sc ii}}
\newcommand{\oi}{O~{\sc i}}
\newcommand{\nii}{N~{\sc ii}}
\newcommand{\oiii}{O~{\sc iii}}
\newcommand{\mgii}{Mg~{\sc ii}}
\newcommand{\tco}{{$^{13}$CO}}
\newcommand{\CO}{{$^{12}$CO}}
\newcommand{\Tco}{{$^{12}$CO}}
\newcommand{\co}{C{$^{18}$O}}
\newcommand{\Lsol}{L$_{\odot}$}
\newcommand{\Msol}{M$_{\odot}$}
%\newcommand{\C18o}{C$^{18}$O($1\rightarrow 0$)}
%\newcommand{\Check}{{\bf ???}}
\newcommand{\mum}{\ensuremath{\mu \mathrm{m}}}
\newcommand{\flux}{flux density}
\newcommand{\solar}{\ensuremath{\odot}}

\newcommand{\epsi}{\varepsilon}

\newcommand{\bgpsarea}{150}
\newcommand{\bcamfwhm}{31.2\arcsec}
\newcommand{\ncores}{$10^4$}
\newcommand{\bgpsdepthlow}{20}
\newcommand{\bgpsdepthhigh}{50}

\newcommand{\TBD}{{\bf TBD}}

\def\Figure#1#2#3#4{
\begin{figure}[htb]
\epsscale{#4}
\plotone{#1}
\caption{#2}
\label{#3}
\end{figure}
}

\def\Table#1#2#3#4#5{
\begin{deluxetable}{#1}
\tablewidth{0pt}
\tablecaption{#2}
\tablehead{#3}
\startdata
\label{#4}
#5
\enddata
\end{deluxetable}
}

\newcommand{\cii}{[C{\sc ii}]}
\def\hi{H{\sc i}}
\newcommand{\smg}{SMG}
\newcommand{\smgs}{SMGs}

\newcommand{\penn}{1}
\newcommand{\casa}{2}
\newcommand{\cso}{3}
\newcommand{\utexas}{4}
\newcommand{\virginia}{5}
\newcommand{\ubc}

  % End definitions


\shorttitle{Measuring galaxy clustering with FIR line intensity mapping}
\shortauthors{}

\begin{document}


%\altaffiltext{\penn}{University of Pennsylvania, Philadelphia, PA 19104}

\begin{abstract}


\end{abstract}

\keywords{
ISM: - molecular clouds --
stars: formation -- high mass
millimeter continuum
}

\section{Introduction}


Atomic (Visbal et al 2011, Gong et al 2012, etc.) and molecular (Lidz et al 2011, etc.) transitions -- such as the 21 cm spin flip transition from H$^{\mathrm{o}}$, CO (2-1), and [CII] 158$\mu$m -- have been investigated as candidates for intensity mapping experiments during the Epoch of Reionization. Of these, the neutral hydrogen case is undoubtedly the most developed in terms of its standing in the literature (cf. Furlanetto, Oh, and Briggs for a review) and in the experimental arena (e.g., PAPER (Parsons et al 2010), MWA), and so interest in measuring the [CII] power spectrum, for instance, has primarily erupted as a means to complement the 21 cm studies at high redshift via the cross-correlation.%\begin{itemize}

A benefit of intensity mapping over individually resolving objects is that the power spectrum is sensitive to the low luminosity galaxies that are below the detection threshold of current and future instruments. Looking forward to EoR, intensity mapping might be the only means available to observe the large population of faint galaxies responsible for reionizing the IGM, given that a telescope such as JWST will only be able to resolve down to 50 percent of UV LF at z ~ 6 (Salvaterra et al 2011). To consider an example at lower redshifts that is \emph{not} intensity mapping but \emph{does} illustrate the potential of power spectra to constrain astrophysically interesting quantities, the SPIRE instrument, even, aboard Herschel misses roughly half of the sources that comprise the cosmic infrared background (CIB) at $z = X.X$ (Bethermin et al 2012). In this case, authors exploited a statistical analysis via measuring the clustering term in the angular power spectrum to uncover information about the sources that, though unresolved, nevertheless contribute to the intensity fluctuations in the SPIRE 250, 350, and 500 $\mu$m wavebands (Amblard et al 2011). From their analysis, which included a halo model approach to describing the clustering power, they were able to estimate such values relevant to galaxy evolution as characteristic mass for most efficient star formation in a host dark matter halo. With a larger survey area and an added framework to tie galaxy luminosity to the halo model, the clustering amplitude of the CIB power as measured by Herschel was used in conjunction with observed number counts above 0.1 mJy to fit various parameters (Viero et al 2012), including again the halo mass which is most efficient for hosting star formation. Despite the novelty of their approach to interpret the CIB power spectrum, Viero et al 2012 encountered significant difficulty in constraining the parameters of their model with existing data. Indeed, Penin et al (2011) explored the ability of the clustering information from CIB angular power spectra to constrain halo model parameters, and found that, while constraints are somewhat improved, simultaneously fitting for clustering data and number counts do not break the degeneracies among the halo model parameters. The authors there and in Viero et al cite unknown redshift distributions of the sources comprising the CIB as the main source of uncertainty in their analyses.

Intensity mapping, on the other hand, contains inherent redshift information encoded along the spectral dimension of the survey volume, and is thus poised to be a valuable complement to current studies of the clustering---and its evolution---of dusty star-forming galaxies at moderate redshift.

Here we examine the value of measuring power spectra of fine structure IR emission lines, including [CII]158$\mu$m, [OI]63$\mu$m, [OIII]88$\mu$m, and [SiII]35$\mu$m, at low to moderate redshifts, specifically between $z = 0.5$ and $z = 3$. As a precursor to Reionization-era experiments, the appeal as a proof-of-principle is obvious, but we focus in this paper on the ability of FIR line intensity mapped power spectra to measure the clustering amplitude of star-forming galaxies. The organization of this paper is as follows. We have calculated the mean emissivity for a suite of IR emission lines based on the IR luminosity function (Bethermin et al 2011) and empirical line to IR luminosity correlations described by Spinoglio et al (2012), and present these results in the context of a power spectrum model in Section 2. In Section 3, we envision suitable platforms---namely the SAFARI instrument aboard future space mission SPICA for the short wavelength lines and a balloon-based experiment for [CII]---for conducting the IR intensity mapping and discuss the feasibility of measuring the power spectra with error bar estimates. We also compare intensity mapping to the approach of individual line detections in order to assess its value 
of probing an otherwise undetected population of galaxies. Finally, in Section 4, we present preliminary results on the ability of IR line intensity mapped power spectra to discriminate between halo models. 




\section{Setting up predictions for IR line power Spectra during $0.5 < z < 3$}

Traditional methods for measuring the spatial autocorrelation of galaxies through galaxy surveys rely on the knowledge of the redshift distribution of sources in the survey. Furthermore, they estimate the true three dimensional clustering of galaxies via the angular projection. Intensity mapping, however, contains intrinsic redshift information and provides a direct measure of the clustering power spectrum in three-dimensional k-space, which makes it a highly complementary probe of structure in the cosmic web. 

The complete auto power spectrum of a given FIR line, $X_i$, as a function of wavenumber $k$, $P_{i,i}(k,z)$, can be separated into power from the clustering of galaxies, $P_{i,i}^{clust}(k,z)$ and a Poisson term describing their discrete nature, $P_{i,i}^{shot}(k, z)$. We compute the full nonlinear matter power spectrum, $P_{nl}(k, z)$, using the publicly available code HALOFIT+, which has been the standard tool for predicting matter power spectra upon its success in fitting state-of-the-art dark matter simulations over a decade ago (Smith et al 2003) . (We note in passing, however, that since that time, authors (cf., e.g., Takahashi et al 2012) have pointed out improvements to the halo model fit on the small scales previously inaccessible due to constraints on simulation resolution.) The clustering component of the [CII] power spectrum is then written as


\begin{equation}
 P_{i,i}^{clust}(k, z) = \bar{I}_{\mathrm{[i]}}^2(z) \bar{b}^2(z) P_{nl}(k, z).
 \end{equation}

Here we have implicitly assumed that the fluctuations in $X_i$ emission trace the matter power spectrum with some average bias, $\bar{b}(z)$. The mean line intensity, $\bar{I}_{i}}(z)$, in units of Jy sr$^{-1}$, can be calculated as

\begin{equation}
\bar{I}_{i}(z) = \int{\mathrm{d}n_{i} \frac{L_{i}}{4\pi D_L^2}} y_i D_A^2 ,
\end{equation}

where the integration is taken with respect to $n_{i}$, the number of galactic $X_i$ emitters per cosmological comoving volume element. (The factor $y_i$ is the derivative of the comoving radial distance with respect to the observed frequency, i.e. $y = d\chi/d\nu = \lambda_{i,rest} (1+z)^2/H(z)$, and $D_A$ is the comoving angular distance.)

Finally, the shot noise component of the total intensity mapped power spectrum---with the same units as the clustering term, namely, Jy sr$^{-1}$(Mpc h$^{-1})^{3}$---takes the form 

\begin{equation}
P_{i,i}^{shot}(k) = \int{\mathrm{d}n_{i} \left( \frac{L_{i}}{4\pi D_L^2} \right)^2 \left( y_i D_A^2 \right)^2}.
\end{equation}



\begin{figure}[h]
\centering
\subfloat(a){
\label{}
\includegraphics[width=0.4\textwidth]{b11_spinoglio_intensity_jysr_vs_z.pdf}}
\subfloat(b){
\label{fig:bethermin_fir_frac}
\includegraphics[width=0.4\textwidth]{intensity_vs_lambda_vs_Hz.pdf}}
\centering
\caption{Intensity of fine structure line emission plotted versus redshift (\emph{top}) and observed wavelength (\emph{bottom}) as predicted from Spinoglio line luminosity fits as applied to the Bethermin (2011) luminosity function.}
\end{figure}

\section{Observational Strategy}

Include plot of fraction of emissivity vs $L_{IR,min}$ to further motivate intensity mapping observation of clustering.
\begin{figure}[h]
\centering
\subfloat(a){
\label{}
\includegraphics[width=0.4\textwidth]{fraction_cii_emissivity_vs_LIRmin_vs_z.pdf}}
\subfloat(b){
\label{fig:bethermin_fir_frac}
\includegraphics[width=0.4\textwidth]{fraction_oi63_emissivity_vs_LIRmin_vs_z.pdf}}
\subfloat(b){
\label{fig:bethermin_fir_frac}
\includegraphics[width=0.4\textwidth]{fraction_SiII35_emissivity_vs_LIRmin_vs_z.pdf}}
\centering
\caption{Fraction of emissivity recovered by integrating the luminosity function using various minimum $L_{IR}$ values. The upper limit for integration is fixed at 10$^{13}$ L$_{\odot}$.}
\end{figure}

\subsection{[CII] power spectrum}

Atmospheric transmission permits uninterrupted spectral coverage for a balloon experiment in the wavelength range 240 to 420 $\mu$m, or $z = 0.5-1.5$ for [CII]158. For concreteness in our discussion of the detectability of the [CII] power spectrum, we sketch the envisioned balloon-borne experiment with various aperture diameters, $D_{ap} = 1.0$ and 3.0 m, and survey  areas, $A_{survey} = 0.1, 1.0$, and 10.0 deg$^2$. Relevant parameters for the experimental platforms envisioned here are summarized in Table 1. 

Predictions---as computed from the method of combining the cosmological matter power spectrum and the IR LF model outlined in Section 2.1---for the  [CII] power spectrum at four redshifts $z = 0.63, 0.88, 1.16$, and $1.48$ in the above redshift range are displayed in Figure 4 for the fiducial case a 3 meter aperture, 1.0 deg$^2$ survey area, and total observing time, $t_{obs}^{survey} = 200$ hours. (Note that we use  $\Delta_{[CII]}^2 = k^3 P_{[CII], [CII]}(k)/(2\pi^2)$ when plotting the power spectrum, where the integral of $\Delta_{[CII]}^2$ over logarithmic k bins is equal to the variance in real space.) At these redshifts, respectively, the average linear bias has been assumed to be $\bar{b} = 2.0, 2.3, 2.6$, and $2.9$, in line with theoretical predictions. We calculate error bar estimates and the mean signal to noise ratio (SNR) for each power spectrum by assuming a spectrally flat noise power spectrum, so that the noise power in each pixel, $P_{N}$, is calculated from

\begin{equation}
P_N = \sigma_N^2 \frac{V_{pix}}{t_{obs}^{pix}},
\end{equation}

where $\sigma_N^2$ is the instrument sensitivity (NEI, in units of Jy sr$^{-1}$ Hz$^{-1/2}$, $V_{pix}$ is the volume of a pixel, and $t_{obs}^{pix}$ is the time spent observing on a single pixel. The variance of a measured $k$, $\sigma^2(k)$, is then written as

\begin{equation}
\sigma^2(k) = \frac{\left({P_{[CII],[CII]}(k) + P_N(k)}\right)^{2}}{N_{mode}},
\end{equation}

where $N_{mode}$ is the number of wavemodes that are sampled for a given $k$ bin of some finite width $\Delta$log(k). (We have chosen $\Delta$log(k) = 0.3 for this analysis.)

The k-averaged SNR, in turn, is calculated from the expression 

\begin{equation}
SNR = \sqrt{\sum_{bins} \left(\frac{P_{[CII],[CII]}(k)}{\sigma(k)}\right)^2}
\end{equation}

Note that It is possible to rewrite $P_N$ in terms of the parameters from Table 1, giving 

\begin{equation}
\begin{split}
P_N& = \sigma_N^2 A_{pix}\Delta r_{los}^{pix} /  {\frac{t_{obs}^{survey}}{n_{beams}/N_{instr}^{spatial}}}\\
& = \sigma_N^2 A_{pix}\Delta r_{los}^{pix} /  \frac{t_{obs}^{survey} N_{instr}^{spatial}}{A_{survey}/A_{pix}}\\
& = \sigma_N^2 \frac{\Delta r_{los}^{pix} A_{survey}}{t_{obs}^{survey} N_{instr}^{spatial}}
\end{split}

\end{equation}

In this form, it becomes apparent that---with fixed number of spatial pixels, spectral resolution, and total observing time---the only factor driving up the amplitude of noise power is the survey area; the effect of increasing aperture only allows access to higher wavenumbers. This behavior is shown clearly in Figure 5, where the SNR is plotted as a function of $k$ for different survey geometries and both mirror diameters. Also seen in Figure 5, the greater number of wavemodes sampled with the larger survey area does not necessarily compensate for the the increase in $P_{N}$.

\begin {table*}[h]
\begin{center}
\caption {Experimental Parameters for Envisioned Balloon Experiment at $z$ = 0.88} \label{tab:title} 
\begin{tabular}{ l c c c c c c }
\hline \hline
 $t_{obs}^{survey}$ (hr)&\multicolumn{6}{c}{200}\\
 $I_{[CII]}$ (Jy sr$^{-1}$)&\multicolumn{6}{c}{6.27 $\times$ 10^3}\\
 NEI (Jy sr$^{-1}$ sec$^{1/2}$)&\multicolumn{6}{c}{2.17 $\times$ 10$^7$}\\
 $B_{\nu}$ (GHz)&\multicolumn{6}{c}{945-1,086}\\
  $\delta_{\nu}$ (GHz)&\multicolumn{6}{c}{2.25}\\
  \hline
 $A_{survey}$ (deg$^2$)&\multicolumn{2}{c}{0.1}&\multicolumn{2}{c}{1.0}&\multicolumn{2}{c}{10.0}\\
 $V_{survey}$ (Mpc^3 h$^{-3}$)&\multicolumn{2}{c}{2.87 $\times$ 10^4}&\multicolumn{2}{c}{2.87 $\times$ 10^6}&\multicolumn{2}{c}{2.87 $\times$ 10^8}\\
 \hline
 $D_{ap}$ (m) & 1 & 3 & 1 & 3 & 1 & 3 \\
 Beam FWHM (arcmin) & 0.42 & 1.25 & 0.42 & 1.25 & 0.42 & 1.25 \\
 $V_{voxel}$ (Mpc^3 h$^{-3}$) &  19.76 & 2.20 & 19.76 & 2.20 & 19.76 & 2.20 \\
 $t_{obs}^{voxel}$ (hr) & 2.0 $\times$ 10^2 & 2.3 $\times$ 10^1 & 2.1 & 2.3 $\times 10^{-1}$& 2.1 $\times 10^{-2}$ & 2.3 $\times 10^{-3}$\\
 $P_N^{voxel}$ (10$^{10}$ Jy$^2$ sr$^{-2}$ Mpc$^3$ h$^{-3}$)  & 1.17 & 1.17 & 117 & 117 & 11, 700 & 11, 700\\
 \hline

\end{tabular}
\end{center}
\end{table}



\begin{figure*}[h]
\centering
\begin{tabular}{cc}
\includegraphics[width=0.4\textwidth]{pcii_z63_halofit_bethermin_spinoglio_lirmax_ap3m_1sqdeg_uhp_obs.pdf} &
\includegraphics[width=0.4\textwidth]{pcii_z88_halofit_bethermin_spinoglio_lirmax_ap3m_1sqdeg_uhp_obs.pdf} \\
\includegraphics[width=0.4\textwidth]{pcii_z116_halofit_bethermin_spinoglio_lirmax_ap3m_1sqdeg_uhp_obs.pdf} &
\includegraphics[width=0.4\textwidth]{pcii_z148_halofit_bethermin_spinoglio_lirmax_ap3m_1sqdeg_uhp_obs.pdf}
\end{tabular}
\caption{\label{} Predicted [CII] power spectra with error estimates from $z$ = 0.63 to $z$ = 1.48 for telescope with 3 meter aperture and a survey area of 1 square degree. Blue, green, magenta, and cyan curves denotes power spectra computed with upper limits of $L_{IR} = 10^{13}, 10^{12}, 10^{11}$, and 10$^{10}$ L$_{\odot}$, respectively.}
\end{figure}

\begin{figure}[h]
\includegraphics[width=0.5\textwidth]{kparkperphist_ICarIS_ap3m_1sqdeg_z88.pdf}
\caption{\label{} Sample histogram of $k_{par}$-$k_{perp}$ for ICarIS [CII] power spectrum at $z = 0.88$, $D_{ap} = 3.0$m and $A_{survey} = 1.0$ deg$^2$.}
\end{figure}

\begin{figure*}[h]
\centering
\begin{tabular}{cc}

\includegraphics[width=0.5\textwidth]{pcii_z63_halofit_bethermin_spinoglio_witherror_constlogk_ICarISkmodes_ap1m_1sqdeg_uhp_obs.pdf} &
\includegraphics[width=0.5\textwidth]{pcii_z88_halofit_bethermin_spinoglio_witherror_constlogk_ICarISkmodes_ap1m_1sqdeg_uhp_obs.pdf} &
\includegraphics[width=0.5\textwidth]{pcii_z116_halofit_bethermin_spinoglio_witherror_constlogk_ICarISkmodes_ap1m_1sqdeg_uhp_obs.pdf} &
\includegraphics[width=0.5\textwidth]{pcii_z148_halofit_bethermin_spinoglio_witherror_constlogk_ICarISkmodes_ap1m_1sqdeg_uhp_obs.pdf} &
 \end{tabular}
 \caption{Predicted [CII] power spectra  from $z$ = 0.63 to $z$ = 1.48 for telescope with 1 meter aperture and a survey area of 1 square degree.}
\end{figure}

\begin{figure*}[t]
\centering
\begin{tabular}{cc}
\includegraphics[width=0.4\textwidth]{snr_vs_kbin_1m_3m_asurveys_uhp_obs_z88.pdf} &
\includegraphics[width=0.4\textwidth]{nmode_vs_kbin_1m_3m_asurveys_uhp_obs_z88.pdf} \\
\end{tabular}
\caption{\label{} Signal to noise on the dimensionless [CII] power spectrum $\Delta_{\textrm{[CII]}}^2$ and number of modes as a function of $k$. The blue lines represent S/N for the 1 meter aperture, red denotes a 3 meter aperture, and purple shows where the two overlap. Survey areas of 0.1, 1, 10 square degree fields are shown as the solid, dashed, and dotted lines, respectively. The special case of a line scan survey with dimensions 1 degree by 1 beam for a 3 meter aperture is plotted as the dash-dotted line.}
\end{figure}


\begin{figure*}[b]
\centering
\begin{tabular}{cc}
\includegraphics[width=0.5\textwidth]{pSiII_z3_bethermin_spinoglio_SPICA_p5sqdeg_R450_dz_0p4_tobs450hr.pdf} &
\includegraphics[width=0.5\textwidth]{pSiII_z3_bethermin_spinoglio_SPICA_1sqdeg_R450_dz_0p4_uhp_obs_tobs1d3hr_lirmax} 
 \end{tabular}
 \caption{Predicted [SiII]35 power spectra at $z$ = 3 for SPICA-SAFARI and survey area $A_{survey}$ = 0.5 deg$^2$ and 1.0 deg$^2$ with  $t_{obs}^{survey} = 450$ hr and 1,000 hr, resp.}
\end{figure}

\subsection{[OI]63, [SiII]35, and other IR line power spectra}

text


\begin {table*}[h] 
\begin{center}
\caption {Experimental Parameters for SPICA-SAFARI Survey of [OI]63 at $z$ = 1.5} \label{tab:title}
\begin{tabular}{ l c c c c c c  }
\hline \hline
$k_{fund}$ (h Mpc$^{-1}$) & \multicolumn{2}{c}{0.03} & \multicolumn{2}{c}{0.05} & \multicolumn{2}{c}{0.07} \\
\hline
NEI (10$^6$ Jy sr$^{-1}$ sec$^{1/2}$) & \multicolumn{2}{c}{6.2} & \multicolumn{2}{c}{6.2} & \multicolumn{2}{c}{6.2} \\
$A_{survey}$ (deg$^2$) & \multicolumn{2}{c}{44} & \multicolumn{2}{c}{16} & \multicolumn{2}{c}{8.1} \\
 $r_{perp}$ (Mpc h$^{-1}$) & \multicolumn{2}{c}{360} & \multicolumn{2}{c}{220} & \multicolumn{2}{c}{160} \\  
$B_{\nu}$ (GHz) & \multicolumn{2}{c}{360} & \multicolumn{2}{c}{220} & \multicolumn{2}{c}{160} \\
$\Delta \nu$ (GHz) & \multicolumn{2}{c}{4.25} & \multicolumn{2}{c}{4.25} & \multicolumn{2}{c}{4.25} \\
$n_{beams}$ (10$^6$ beams) & \multicolumn{2}{c}{3.3} & \multicolumn{2}{c}{1.1} & \multicolumn{2}{c}{0.61} \\
\hline
$t_{obs}^{survey}$ (hr) & 450 & 1,000 & 450 & 1,000 & 450 & 1,000 \\
\hline
$t_{obs}^{pix}$ (sec) & 40 & 88 & 110 & 240 & 220 & 480 \\
$P_N$ (10$^{10}$ Jy$^2$ sr$^{-2}$ Mpc$^3$ h$^{-3}$) & 12 & 5.3 & 4.2 & 1.9 & 2.2 & 0.97 \\
\hline
\end{tabular}
\end{center}
\end{table}

\begin{figure}
\includegraphics[width=0.5\textwidth]{poi63_z1p5_SPICA_bethermin_spinoglio_p5sqdeg_R450_dz0p4_tobs450hr_lirmax.pdf}
\caption{Predicted [OI]63 power spectra at $z$ = 1.5 for SPICA-SAFARI and survey area $A_{survey}$ = 0.5 deg$^2$ with  $t_{obs}^{survey} = 450$ hr.}
\end{figure}

\begin{figure}
\includegraphics[width=0.5\textwidth]{snr_vs_k_vs_kfund_oi63_SPICA_uhp_z148}
\caption{}
\end{figure}

\begin{figure}
\includegraphics[width=0.5\textwidth]{nmode_vs_k_vs_kfund_SPICA_uhp_z148.pdf}
\caption{}
\end{figure}


\begin{figure*}
\centering
\subfloat(a){
\label{fig:bethermin_sfrd_frac}
\includegraphics[width=0.4\textwidth]{bethermin_sfrd_frac.pdf}}

\subfloat(b){
\label{fig:bethermin_fir_frac}
\includegraphics[width=0.4\textwidth]{bethermin_fir_frac.pdf}}

\subfloat(c){
\label{fig:bethermin_cii_frac_rconst}
\includegraphics[width=0.4\textwidth]{bethermin_cii_fraction_rconst.pdf}}

\subfloat(d){
\label{fig:bethermin_lciilfir_ratio_rconst}
\includegraphics[width=0.4\textwidth]{bethermin_lciilfir_ratio_rconst.pdf}}

\centering
\caption{\label{fig:multi_panel} Redshift evolution of the star formation rate (a), FIR luminosity (b), [CII] luminosity (c), and $[CII]-L_{FIR}$ relation (d) based on the Bethermin et al (2011) IR luminosity function. In panels (a)-(c), the red, green, and blue colors denote contributions to the luminosity function by ULIRGs, LIRGs, and normal galaxies, respectively, and black lines denote a total. The bottom two panels incorporate two prescriptions for finding $L_{[CII]}$: (1) $L_{[CII]}$ as a function of $L_{IR}$ from Spinoglio et al (2012) (\emph{solid lines}) and (2)  $L_{CII} = 0.003 \times L_{FIR}$ (\emph{dashed lines}). In panel (d), the red, green, and blue line represent the mean $[CII]-L_{FIR}$ for their respective classes. Note there is no externally imposed redshift evolution on the [CII] luminosity in all cases.}
\end{figure*}


\section{Cross Power Spectrum}

Visbal and Loeb (2010) showed how the cross spectra can be used to differentiate between a target line and a contaminating line (or "bad line", in their words). The cross power spectrum of two distinct lines can generally be written

\begin{align}
P_{i,j}(k)& = \bar{S_i} \bar{S_j} \bar{b_i} \bar{b_j} P_{lin}(k) + P_{shot}^{i,j}(k) %\\
\end{align}

\begin{figure}[h]
\centering
\includegraphics[width=0.4\textwidth]{pciinii122_z1_halofit_bethermin_spinoglio_witherror_constlogk_ICarISkmodes_ap3m_1sqdeg_uhp_obs}
\caption{Predicted cross power spectrum $P_{[CII]-[NII]}$ at $z = 1$ for $D_{ap}$= 3.0 and $A_{survey}$ = 1.0 deg$^2$. }
\end{figure}



\begin{center}[h]
	\begin{tabular}{ l  c  c }
	\hline
	 & ICarIS & SPICA-SAFARI \\
	\hline \hline
	Target Line & [CII]158$\mu$m & [OI]63$\mu$m \\
	\hline
	Redshift & 1.5 & 1.5 \\
	\hline
	Aperture (m) & 3.0 & 3.0 \\
	\hline
	$A_{survey}$ (deg$^2$) & 1 & 1\\ 
	\hline
	$R$ & 450 & 2000 \\
	\hline
	$\lambda_{center}$ ($\mu$m) & 393 & 160 \\
	\hline
	$\theta_{FWHM}$ ('') & 33 & 13 \\
	\hline
	Line Sensitivity (W m$^{-2}$ sec$^{1/2}$) & 7.1 $\times 10^{-18}$ & 2.0 $\times 10^{-19}$ \\
	\hline
	NEI (Jy sr$^{-1}$ sec$^{1/2}$ ) & 1.0 $\times$ 10^7 & 1.3 $\times$ 10^7 \\
	\hline
	$V_{voxel}$ (Mpc$^3$ h$^{-3}$) & 5.8 & 0.53 \\
	\hline
	Bandwidth (GHz) & 108 & 350* \\
	\hline 
	$\Delta\nu$ (GHz) & 1.7 & 0.94\\
	\hline
	$t_{obs}^{survey}$ (hr) & 200 & 1000 \\
	\hline
	$t_{obs}^{voxel}$ (hr) & 0.4 & 1.1 \\
	\hline
	$N_{beam}$ (deg$^{-2}$) & 11,926 & 71,951 \\
	\hline
	signal (Jy sr$^{-1}$) & 2,593 & 1,373 \\
	\hline
	$P_{noise}$ (Jy$^2$ sr$^{-2}$ Mpc$^{-3}$ h$^{3}$) & 4.0 $\times 10^{11}$ & 9.6 $\times 10^{10}$ \\ %3.5 $\times 10^9$ \\
	\hline
	$N_{modes}^{obs}$ & 3.7 $\times$ 10^5 & 1.3 $\times$ 10^7 \\
	\hline
	SNR(k = 0.16 h Mpc$^{-1}$) & 5.4 & 3.1 \\ %14 \\
	\hline
	SNR(k =  2.6 h Mpc$^{-1}$) & 24 &130 \\
	\hline
	\end{tabular}
\end{center}
	





\bibliography{/home/jaguirre/Latex/references}

\end{document}

